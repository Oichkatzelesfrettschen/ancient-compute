%% Appendix C: Performance Data and Graphs

\section*{Operation Timing Analysis}

\subsection*{Execution Time by Operation Type}

The following graph shows execution times for all 32 operations on the logarithmic scale:

\begin{figure}[h]
    \centering
    \begin{tikzpicture}
        \begin{axis}[
            xlabel=Operation Type,
            ylabel=Execution Time (seconds, log scale),
            ymode=log,
            ymin=1,
            ymax=1000,
            xtick=data,
            grid=major,
            width=0.95\textwidth,
            height=6cm,
            tick label style={font=\tiny},
            x tick label style={rotate=45, anchor=north east},
            legend pos=upper left,
            legend style={font=\small}
        ]
        
        % Arithmetic operations (ADD, SUB: 8s; MUL: 400s; DIV: 750s)
        \addplot[fill=blue!50, bar width=10pt] coordinates {
            (ADD,8)
            (SUB,8)
            (MUL,400)
            (DIV,750)
            (SQRT,250)
            (CMP,10)
        };
        
        % Control flow (JMP, JZ, etc.: 3s)
        \addplot[fill=green!50, bar width=10pt] coordinates {
            (JMP,3)
            (JZ,3)
            (JNZ,3)
            (JLT,3)
            (JGT,3)
        };
        
        % Memory (LOAD, STOR: 15s)
        \addplot[fill=yellow!50, bar width=10pt] coordinates {
            (LOAD,15)
            (STOR,15)
        };
        
        % I/O (RDCRD, WRPCH: 30s; WRPRN: 2.5s)
        \addplot[fill=red!50, bar width=10pt] coordinates {
            (RDCRD,30)
            (WRPCH,30)
            (WRPRN,2.5)
            (CHKS,5)
        };
        
        \legend{Arithmetic, Control, Memory, I/O}
        \end{axis}
    \end{tikzpicture}
    \caption{Execution time for all operations (logarithmic scale). DIV dominates at 750 seconds.}
    \label{fig:timing-log}
\end{figure}

\subsection*{Performance Statistics}

Summary statistics for common operations:

\begin{table}[h]
    \centering
    \small
    \begin{tabular}{lrrrr}
    \toprule
    \textbf{Operation} & \textbf{Time (s)} & \textbf{Rate (ops/sec)} & \textbf{Rate (ops/hour)} & \textbf{Category} \\
    \midrule
    WRPRN & 2.5 & 0.40 & 1,440 & I/O \\
    ADD & 8.0 & 0.125 & 450 & Arithmetic \\
    SUB & 8.0 & 0.125 & 450 & Arithmetic \\
    CMP & 10.0 & 0.10 & 360 & Arithmetic \\
    LOAD & 15.0 & 0.067 & 240 & Memory \\
    STOR & 15.0 & 0.067 & 240 & Memory \\
    RDCRD & 30.0 & 0.033 & 120 & I/O \\
    WRPCH & 30.0 & 0.033 & 120 & I/O \\
    SQRT & 250.0 & 0.004 & 14 & Arithmetic \\
    MUL & 400.0 & 0.0025 & 9 & Arithmetic \\
    DIV & 750.0 & 0.0013 & 5 & Arithmetic \\
    \midrule
    \textbf{Average} & \textbf{$\sim$100} & \textbf{$\sim$0.01} & \textbf{$\sim$40} & \\
    \bottomrule
    \end{tabular}
    \caption{Performance metrics: operations per second and per hour (assuming sustained operation).}
    \label{tab:performance-metrics}
\end{table}

**Key insight**: Print operation (WRPRN, 2.5s) is fastest; division (750s) is slowest. Average operation time: $\sim$100 seconds.

\subsection*{Throughput Comparison}

How does the Babbage engine compare to modern computers?

\begin{table}[h]
    \centering
    \small
    \begin{tabular}{llll}
    \toprule
    \textbf{System} & \textbf{Architecture} & \textbf{Operations/sec} & \textbf{Ratio (Babbage = 1×)} \\
    \midrule
    Babbage (average ADD) & Mechanical decimal & 0.125 & 1.0× \\
    Babbage (worst DIV) & Mechanical decimal & 0.0013 & 0.01× \\
    IBM 650 (1954) & Vacuum tube decimal & 100 & 800× \\
    IBM System/360 (1964) & Transistor binary & 1,000,000 & 8,000,000× \\
    Intel Pentium (1994) & Binary, 100 MHz & 100,000,000 & 800,000,000× \\
    Modern CPU (2025, 5 GHz) & Binary, multi-core & 10,000,000,000 & 80,000,000,000× \\
    \bottomrule
    \end{tabular}
    \caption{Performance comparison: Babbage engine vs. historical and modern computers.}
    \label{tab:performance-comparison}
\end{table}

The Babbage engine is approximately **100 million to 100 billion times slower** than modern CPUs, but all three compute the same algorithms—demonstrating substrate independence.

\subsection*{Equivalent Performance Metrics}

If we normalize to modern metrics:

\[
\text{Babbage equivalent MIPS} = 0.125 \text{ operations/sec} \times 0.001 \text{ (to MIPS)} = 0.000125 \text{ MIPS}
\]

Or: **1.25 × 10^{-4} MIPS** (milliinstructions per second).

For reference:
\begin{itemize}
    \item Original IBM PC (1981): 0.33 MIPS
    \item Pentium (1994): 100 MIPS
    \item Modern multicore: 100,000--500,000 MIPS
\end{itemize}

\subsection*{Cost-Performance Analysis}

Cost per operation:

\[
\text{Cost/op} = \frac{55,000 \text{ GBP}}{0.125 \text{ ops/sec} \times 3,600 \text{ sec/hour} \times 24 \text{ hours/day} \times 365 \text{ days/year}} = \frac{55,000}{3,942,000} \approx 0.014 \text{ GBP/op}
\]

Or: **140 million GBP per MIPS** (compared to modern: $< 0.01 GBP per MIPS).

Cost per component: 55,000 GBP ÷ 40,000 components = **1.38 GBP/component** (in 1910s currency).

\subsection*{Energy Consumption}

Mechanical systems have minimal electrical consumption (card reader/punch motors only). Estimate:

\begin{itemize}
    \item Hand crank operation: 50--100 W mechanical power input
    \item Steam engine (1--2 HP): 750--1,500 W mechanical power
    \item Electrical for card equipment: 500 W (motors, solenoids)
\end{itemize}

Total operational power: **1--2 kW** (mechanical + electrical), vastly less than modern computers (100--1,000 W CPU alone).

Energy cost per operation: negligible (mechanical friction dominates).

\subsection*{Memory System Performance}

Access time characteristics:

\begin{table}[h]
    \centering
    \small
    \begin{tabular}{lll}
    \toprule
    \textbf{Operation} & \textbf{Time} & \textbf{Bandwidth} \\
    \midrule
    LOAD single word & 15 sec & 3.3 words/sec \\
    STOR single word & 15 sec & 3.3 words/sec \\
    Bulk load (100 words) & $\sim$1,500 sec & 0.067 words/sec \\
    Card read (I/O) & 30 sec & 2 words/sec \\
    Card write (I/O) & 30 sec & 2 words/sec \\
    \bottomrule
    \end{tabular}
    \caption{Memory and I/O bandwidth analysis.}
    \label{tab:memory-bandwidth}
\end{table}

Memory bandwidth is 15 sec × 2,000 address space ÷ 50-digit word = **100,000 bits/sec** (theoretical maximum with random access).

Actual throughput limited by mechanical latency: **2--3 words/sec sustained** (LOAD or STOR continuous).

